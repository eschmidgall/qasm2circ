%  Time 01: 
%    Gate 00 H(q0) 
%  Time 02: 
%    Gate 01 c-U(q0,q1) 
%  Time 03: 
%    Gate 02 H(q0) 
%  Time 04: 
%    Gate 03 measure(q0) 
%  Time 05: 
%    Gate 04 c-V(q0,q1) 
%  Time 06: 
%    Gate 05 nop(q0) 
%    Gate 06 nop(q1) 

% Qubit circuit matrix:
%% q0: gAxA, gBxA, gCxA, gDxA, gExA, gFxA, N   
% q1: n  , gBxB, n  , n  , gExB, gFxB, n   

\documentclass[11pt]{article}
\input{xyqcirc.tex}

% definitions for the circuit elements 
\def\gAxA{\op{H}\w\A{gAxA}}
\def\gBxA{\b\w\A{gBxA}}
\def\gBxB{\op{U}\w\A{gBxB}}
\def\gCxA{\op{H}\w\A{gCxA}}
\def\gDxA{\meter\w\A{gDxA}}
\def\gExA{\b\W\A{gExA}}
\def\gExB{\op{V}\w\A{gExB}}
\def\gFxA{*-{}\W\A{gFxA}}
\def\gFxB{*-{}\w\A{gFxB}}

% definitions for bit labels and initial states 
\def\bA{ \q{q_{0}}}
\def\bB{ \q{q_{1}}}

% The quantum circuit as an xymatrix 
\xymatrix@R=5pt@C=10pt{ 
    \bA & \gAxA &\gBxA &\gCxA &\gDxA &\gExA &\gFxA &\N  
\\  \bB & \n   &\gBxB &\n   &\n   &\gExB &\gFxB &\n  
% 
% Vertical lines and other post-xymatrix latex % 
\ar@{-}"gBxB";"gBxA"\ar@{=}"gExB";"gExA"}
 
\end{document}