%  Time 01: 
%    Gate 00 nop(q0) 
%    Gate 01 nop(q2) 
%  Time 02: 
%    Gate 02 Z4(q0,q1,q2,q3) 
%  Time 03: 
%    Gate 03 MeasH(q1) 
%    Gate 04 MeasH(q2) 

% Qubit circuit matrix:
%% q0: gAxA, gBxA, n  , n   
% q1: n  , gBxB, gCxB, N   
% q2: gAxC, gBxC, gCxC, N   
% q3: n  , gBxD, n  , n   

\documentclass[11pt]{article}
\input{xyqcirc.tex}

% definitions for the circuit elements 
\def\gAxA{*-{}\w\A{gAxA}}
\def\gAxC{*-{}\w\A{gAxC}}
\def\gBxA{\b\w\A{gBxA}}
\def\gBxB{\b\w\A{gBxB}}
\def\gBxC{\b\w\A{gBxC}}
\def\gBxD{\b\w\A{gBxD}}
\def\gCxB{\dmeter{H}\w\A{gCxB}}
\def\gCxC{\dmeter{H}\w\A{gCxC}}

% definitions for bit labels and initial states 
\def\bA{\qv{q_{0}}{\psi}}
\def\bB{\qv{q_{1}}{+}}
\def\bC{\qv{q_{2}}{+}}
\def\bD{\qv{q_{3}}{\phi}}

% The quantum circuit as an xymatrix 
\xymatrix@R=5pt@C=10pt{ 
    \bA & \gAxA &\gBxA &\n   &\n  
\\  \bB & \n   &\gBxB &\gCxB &\N  
\\  \bC & \gAxC &\gBxC &\gCxC &\N  
\\  \bD & \n   &\gBxD &\n   &\n  
% 
% Vertical lines and other post-xymatrix latex % 
\ar@{-}"gBxD";"gBxA"\ar@{-}"gBxD";"gBxB"\ar@{-}"gBxD";"gBxC"}
 
\end{document}