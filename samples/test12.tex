%  Time 01: 
%    Gate 00 h(q0) 
%  Time 02: 
%    Gate 01 CU2(q0,q1,q2) 
%  Time 03: 
%    Gate 02 h(q0) 
%  Time 04: 
%    Gate 03 CV2(q2,q0,q1) 

% Qubit circuit matrix:
%% q0: gAxA, gBxA, gCxA, gDxA, n   
% q1: n  , gBxB, n  , gDxB, n   
% q2: n  , gBxC, n  , gDxC, n   

\documentclass[11pt]{article}
\input{xyqcirc.tex}

% definitions for the circuit elements 
\def\gAxA{\op{H}\w\A{gAxA}}
\def\gBxB{\gnqubit{U}{d}\w\A{gBxB}}
\def\gBxC{\gspace{U}\w\A{gBxC}}
\def\gBxA{\b\w\A{gBxA}}
\def\gCxA{\op{H}\w\A{gCxA}}
\def\gDxA{\gnqubit{V}{d}\w\A{gDxA}}
\def\gDxB{\gspace{V}\w\A{gDxB}}
\def\gDxC{\b\w\A{gDxC}}

% definitions for bit labels and initial states 
\def\bA{ \q{q_{0}}}
\def\bB{ \q{q_{1}}}
\def\bC{ \q{q_{2}}}

% The quantum circuit as an xymatrix 
\xymatrix@R=5pt@C=10pt{ 
    \bA & \gAxA &\gBxA &\gCxA &\gDxA &\n  
\\  \bB & \n   &\gBxB &\n   &\gDxB &\n  
\\  \bC & \n   &\gBxC &\n   &\gDxC &\n  
% 
% Vertical lines and other post-xymatrix latex % 
\ar@{-}"gBxB";"gBxA"\ar@{-}"gDxB";"gDxC"}
 
\end{document}